\section{Time-of-check to time-of-use (TOC/TOU)}

As the name implies, the TOC/TOU condition arises when a certain value is checked (e.g. to verify that it satisfies a condition) and then it is used to perform some operation. If the \textit{check} and the \textit{use} operations are distinct and do not constitute a single atomic block, it may be possible to manipulate the value after it is checked and before it is used. \\

A typical example for illustrating TOC/TOU vulnerabilities is the following: a privileged program (e.g. a \texttt{Set-UID} program) checks whether the current user\footnote{In this case we are referring to the \textit{real} user, as opposed to the \textit{effective} user which is assumed to be root.} has permissions to access a file. If it does, it opens and writes the file; otherwise it fails with an error message. The vulnerability can be exploited, for instance, by replacing the file in the time interval that occurs between the permission check and the opening of the file.~\citep{setuid} \\

\begin{table}[H]
\centering
\begin{tabular}{|l|l|}
\hline
\thead[c]{\textbf{\texttt{Set-UID} program}} & \thead[c]{\textbf{Attacker}} \\ \hline
\makecell[l]{Check permissions for file \texttt{/tmp/file}} & \\
& \makecell[l]{Symlink \texttt{/etc/passwd} to \texttt{/tmp/file}}  \\
\makecell[l]{Open file \texttt{/tmp/file $\rightarrow$ /etc/passwd}}  & \\
\makecell[l]{Write file \texttt{/tmp/file $\rightarrow$ /etc/passwd}}  & \\ \hline
\end{tabular}
\caption{Exploitation of TOC/TOU vulnerability in a \texttt{Set-UID} program}
\label{tab:setuid}
\end{table}

In the following section we are going to explore a different example, related to a hypothetical banking system.

\subsection{TOC/TOU example~\citep{courseslides}}

Listing~\ref{listing:badbank} shows a PHP program used for handling a money transfer between two customers of the same bank. \textsc{User1} has 500\$ in her account and she wants to transfer 400\$ to \textsc{User2}'s account, which is currently empty.

\begin{listing}[H]
\begin{minted}[startinline, breaklines]{php}
// Check if User1's balance is sufficient
$balance1 = query("SELECT balance FROM users WHERE id = User1");
if ($balance1 < 400) {
    die("Insufficient balance");
}

// Calculate User1's new balance
$balance1 -= 400;

// Store money transfer
query("UPDATE users SET balance = $balance1 WHERE id = User1");
query("UPDATE users SET balance = balance + 400 WHERE id = User2");
\end{minted}
\caption{Example of vulnerable program handling a money transfer}
\label{listing:badbank}
\end{listing}

Assume that \textsc{User1} requests the operation two times in a row (e.g. by clicking the ``Confirm'' button twice on her home banking application), resulting in two (asynchronous) requests being sent to the server. Table~\ref{tab:toctou} represents a possible resulting scenario.

\begin{table}[H]
\centering
\begin{tabular}{|l|l|c|c|}
\hline
\thead[c]{\textbf{First request}} & \thead[c]{\textbf{Second request}} & \thead[c]{\textsc{User1}} & \thead[c]{\textsc{User2}} \\ \hline
\makecell[l]{Check \textsc{User1}'s balance \\ $\hookrightarrow$ \texttt{\$balance1} = 500} & & 500\$ & 0\$ \\
& \makecell[l]{Check \textsc{User1}'s balance \\ $\hookrightarrow$ \texttt{\$balance1} = 500} & &  \\
\makecell[l]{Calculate new balance \\ $\hookrightarrow$ \texttt{\$balance1} = 100} & & & \\
& \makecell[l]{Calculate new balance \\ $\hookrightarrow$ \texttt{\$balance1} = 100} & & \\
\makecell[l]{Store money transfer \\ $\hookrightarrow$ \texttt{\$balance1} = 100} & & 100\$ & 400\$ \\
& \makecell[l]{Store money transfer \\ $\hookrightarrow$ \texttt{\$balance1} = 100} & \textcolor{red}{100\$} & \textcolor{red}{800\$} \\ \hline
\end{tabular}
\caption{Parallel execution of two concurrent money transfer requests. \\ The two rightmost columns show the stored balances for each user}
\label{tab:toctou}
\end{table}

At the end of the process, \textsc{User2} ends up with twice the money he was intended to receive. Notice that the additional money is not taken from \textsc{User1}'s account, but ``created'' out of nothing. Of course the problem resides in the fact that the check of \textsc{User1}'s initial balance and its subsequent update are two distinct (non-atomic) operations.

\subsection{Purposed solution}

A simple (yet not optimal) solution could be to leverage the atomicity of single database queries:

\begin{listing}[H]
\begin{minted}[startinline, breaklines]{php}
// Update User1's balance only if she has at least 400$
query("UPDATE users SET balance = balance - 400 WHERE id = User1 AND balance >= 400");

// Check if the balance was actually updated
if (affected_rows == 0) {
    die("Insufficient balance");
}

// Update User2's balance
query("UPDATE users SET balance = balance + 400 WHERE id = User2");
\end{minted}
\caption{Fixed version of the program from Listing~\ref{listing:badbank}}
\label{listing:fix}
\end{listing}

This solves the problem because the first query checks the balance and updates it in one atomic operation. A better solution would be to wrap both queries in an SQL transaction and commit only at the end of all operations.
