\section{Introduction}

A race condition occurs when multiple concurrent threads try to access and change some shared data without proper synchronization. The resulting change will depend on the order of execution of those threads, and since the operating system can switch between threads at any time, it is not possible to predict what the final result will be.\\

Common examples of shared data are:
\begin{itemize}
  \item \textbf{Regular files} on the filesystem: by default nothing prevents one thread from reading and writing files that are also being used by another thread. Developers may avoid this by using a file locking mechanism
  \item \textbf{Shared memory}: requires using mutexes or semaphores to ensure that critical operations are executed atomically\footnote{An operation is atomic if it is performed as a single, indivisible step without the possibility of interference from other operations.}
  \item \textbf{Databases}. Database Management Systems (DBMS) usually provide built-in support for concurrent connections with the use of transactions\footnote{Notice that this does not mean that operations involving databases are automatically immune to race conditions. On the contrary, as we will see misuse of database transactions is often the cause of race condition vulnerabilities in web applications.}.
\end{itemize}

\subsection{First example}

We will now see one simple kind of bugs that race conditions may cause with a practical example. The Python code in Listing~\ref{listing:ex} reads an integer from a file on the disk, it increments the value by 1, then writes back the result to the file.

\begin{listing}[H]
\begin{minted}[python3]{python}
# Open and read file
with open("myfile", "r") as f:
    value = int(f.read())

# Increment value
value += 1

# Write back new value
with open("myfile", "w") as f:
    f.write(str(value))
\end{minted}
\caption{Example of program affected by a race condition}
\label{listing:ex}
\end{listing}

At a first glance, it seems reasonable to assume that each execution of the program will always result in the value stored in the file being incremented by 1. Therefore, running the program \textit{n} times will result in the value stored in the file being incremented by \textit{n}. If we consider the possibility of running the program in multiple concurrent threads, however, this assumption is no longer guaranteed. In fact, the operating system may switch between threads at any point during the execution, leading to different outcomes. \\

For the sake of this example, suppose that we are running the program in two concurrent threads and let the initial value stored in the file be \texttt{19}. We will explore three of possible outcomes, result of just as many ways the scheduler can switch between threads.

\subsubsection*{Scenario 1 (sequential execution)}

This is the simplest (and luckiest) case. The scheduler does not switch between threads in the middle of the execution. Table~\ref{tab:scenario1} illustrates the order of execution of the relevant operations of each thread in a chronological order from top to bottom.

\begin{table}[H]
\centering
\begin{tabular}{|l|l|}
\hline
\thead[c]{\textbf{Thread \#1}} & \thead[c]{\textbf{Thread \#2}} \\ \hline
\makecell[tl]{Open and read file (\texttt{value = 19})} & \\
\makecell[tl]{Increment value (\texttt{value = 20})} &  \\
\makecell[tl]{Write back new value (\texttt{value = 20})} &  \\
& \makecell[tl]{Open and read file (\texttt{value = 20})} \\
& \makecell[tl]{Increment value (\texttt{value = 21})}  \\
& \makecell[tl]{Write back new value (\texttt{value = 21})}  \\  \hline
\end{tabular}
\caption{Execution order in Scenario 1}
\label{tab:scenario1}
\end{table}

This case is equivalent to running the program sequentially two times with no concurrency. No race conditions arise, and the final value stored in the file is \texttt{21}, that is the original value incremented by 2.

\subsubsection*{Scenario 2 (interleaved execution)}

If the second thread starts executing before the first one has finished writing its result, it will read the ``old'' value from the file.

\begin{table}[H]
\centering
\begin{tabular}{|l|l|}
\hline
\thead[c]{\textbf{Thread \#1}} & \thead[c]{\textbf{Thread \#2}} \\ \hline
\makecell[tl]{Open and read file (\texttt{value = 19})} & \\
\makecell[tl]{Increment value (\texttt{value = 20})} & \\
& \makecell[tl]{Open and read file (\texttt{\textcolor{red}{value = 19}})} \\
\makecell[tl]{Write back new value (\texttt{value = 20})} & \\
& \makecell[tl]{Increment value (\texttt{value = 20})} \\
& \makecell[tl]{Write back new value (\texttt{value = 20})}  \\  \hline
\end{tabular}
\caption{Execution order in Scenario 2}
\label{tab:scenario2}
\end{table}

At the end of the execution, we have lost the update from the first thread. The final value stored in the file is \texttt{20}, that is the original value incremented only by 1.

\subsubsection*{Scenario 3 (interleaved execution)}

Table~\ref{tab:scenario3} illustrates the situation where Thread~\#1 ``wraps'' the execution of Thread~\#2: Thread~\#1 is the first to start and the last to finish.

\begin{table}[H]
\centering
\begin{tabular}{|l|l|}
\hline
\thead[c]{\textbf{Thread \#1}} & \thead[c]{\textbf{Thread \#2}} \\ \hline
\makecell[tl]{Open and read file (\texttt{value = 19})} & \\
& \makecell[tl]{Open and read file (\texttt{value = 19})} \\
\makecell[tl]{Increment value (\texttt{value = 20})} & \\
& \makecell[tl]{Increment value (\texttt{value = 20})} \\
& \makecell[tl]{Write back new value (\texttt{value = 20})}  \\
\makecell[tl]{Write back new value (\texttt{\textcolor{red}{value = 20}})}  & \\ \hline
\end{tabular}
\caption{Execution order in Scenario 3}
\label{tab:scenario3}
\end{table}

This case results in losing the update from the second thread. As in Scenario~2, the final value stored in the file is \texttt{20} instead of \texttt{21}.

\begin{center}
  $\ast$~$\ast$~$\ast$
\end{center}

This example shows how a race condition can affect even the simplest program by making its result dependent on the order of execution of concurrent threads (and therefore practically impossible to predict). The fact that the bug does not always present itself (as in Senario~1) makes it even harder to detect, expecially in more complex situations. \\

In real-world applications race conditions can cause severe vulnerabilities: think about lost or repeated transactions in the context of a banking system. A specific type of race condition that is frequently found both in binary and in web applications is described the following section.
